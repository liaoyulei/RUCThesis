\chapter{一些例子}
\section{定义定理证明}
\begin{defin}~\cite[定义1.33]{Zhou:2016}
设$x\in\mathbb{R}^n$, $E$是$\mathbb{R}^n$中的非空点集, 称
\[d(x,E)=\inf\{|x-y|: y\in E\}\]
为$x$到$E$的距离.
\end{defin}

\begin{lemma}~\cite[定理1.25]{Zhou:2016}
若$E$是$\mathbb{R}^n$中非空点集, 则$d(x,E)$作为$x$的函数在$\mathbb{R}^n$上是一致连续的.
\end{lemma}

\begin{coro}
若$F_1,F_2$是$\mathbb{R}^n$中两个非空闭集且其中至少有一个是有界的, 则存在$x\in F_1,x\in F_2$, 使得
\[|x_1-x_2|=d(F_1,F_2).\]
\end{coro}

\begin{theorem}
若$F$是$\mathbb{R}^n$中的闭集, $f(x)$是定义在$F$上的连续函数, 且$|f(x)|\leq M(x\in F)$, 则存在$\mathbb{R}^n$上的连续函数$g(x)$满足
\[|g(x)|\leq M,\quad g(x)=f(x),\quad x\in F.\]
\end{theorem}

\begin{proof}
证明环境使用样例.
\end{proof}

\begin{remark}
上述定理在$f(x)$无界时也成立.
\end{remark}

\section{图表公式}
图~\ref{fig:logo}为中英校名. 有时候图片自动上浮有时候会遮住页眉线, 可以将{\tt [htbp]}更改为{\tt [H]}将图片固定在当前位置.
\begin{figure}[htbp]
\centering\includegraphics[width=\textwidth]{figures/logo3.pdf}
\caption{中英校名}\label{fig:logo}
\end{figure}

\begin{table}[htbp]
\centering\caption{表格样例}\label{tab:tab}\begin{tabular}{ccccc}
\hline
1 & 2 & 3 & 4 & 5\\
\hline
\multicolumn{3}{c}{三列合并} & 4 & 5\\
\hline
4 & 5 & 6 & 7 & 8\\
\hline
\end{tabular}\end{table}

下面是几个数学公式的例子, 单行公式:
\[c=a-b.\]

带编号公式并引用~\eqref{eq:single}:
\begin{equation}\label{eq:single}
c=a+b.
\end{equation}

使用{\tt cases}环境的大括号公式:
\begin{equation}\begin{cases}
z=x+y,&(x,y)\in\mathbb{R}^2,\\
u=x-y,&(x,y)\in\mathbb{R}^2.
\end{cases}
\end{equation}

多行公式:
\begin{align*}
\psi_i(x,y)=&-\dfrac18(1+x_ix)(1+y_iy)(4+2x_ix+2y_iy-5x^2-5y^2)\\
&-x_i(\psi_{i\bmod4+13}+4\psi_{(i+2)\bmod4+13})+y_i(4\psi_{i+12}+\psi_{(i+1)\bmod 4+13})\\
&+4(\psi_{i+16}-\psi_{(i+2)\bmod4+17})-(\psi_{i\bmod4+17}-\psi_{(i+1)\bmod4+17}).
\end{align*}

多行带编号公式并引用~\eqref{eq:multi}:
\begin{equation}\begin{aligned}\label{eq:multi}
\|v\|_{L^2(\Omega)}=&\left(\int_{\Omega}|v(x)|^2\mathrm{d}x\right)^{1/2}\\
\sim&\lim_{N\to\infty}\left(\dfrac{V}{N}\sum_{k=1}^N|v(x_i)|^2\right)^{1/2}.
\end{aligned}\end{equation}

给多行公式的每一行标号:
\begin{align}
z=&x+y,\\
u=&x+z-y.
\end{align}

\section{参考文献及脚注}
参考文献的{\tt author}域请写作{\tt \{作者1姓\quad 名首字母\} and \{作者2姓\quad 名首字母\}}的形式. 例如{\tt author=\{\{Arnold D N\}\quad and\quad \{Awanou G\}\}}, 主要为了避免英文人名的字母全部大写.

引用样例~\cite{Arnold:2011}, 也可以加入具体信息~\cite[Section 9.1]{Brezis:2011}, 或者同时引用多篇~\cite{Chen:2005,Arnold:2011,Brezis:2011}. 参考文献将自动按照论文中的引用顺序排序.

脚注样例\LaTeX~\footnote{https://zh.wikipedia.org/wiki/LaTeX}.