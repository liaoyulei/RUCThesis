\chapter{引言}
RUCThesis~\footnote{https://github.com/liaoyulei/RUCThesis}主要修正了本科生部分. 请直接使用{\tt ructhesis.cls}文件. 2015年12月正式实施了新的参考文献著录国标, 目前已经使用. 

建议使用TeX Live~\footnote{http://www.tug.org/texlive/}, 其自带编辑器TeXworks, 或者自行安装编辑器TeXstudio~\footnote{http://texstudio.sourceforge.net/}. 这里我们使用XeLaTeX作为引擎, 依次按照{\tt XeLaTeX-BibTeX-XeLaTeX-XeLaTeX}的顺序编译{\tt main.tex}:\\
{\tt
\$ xelatex main.tex\\
\$ bibtex main.tex\\
\$ xelatex main.tex\\
\$ xelatex main.tex\\}

\textbf{重要信息:}
\begin{enumerate}
\item 编辑器要用UTF-8的编码要不然你打开是乱码.
\item 排版引擎使用XeLaTeX, 要不然会报错.
\item 慎重使用CTEX, 大概率无法编译.
\item 请使用{\tt print.tex}将单面和双面打印部分分离. 直接在打印机上选择页码会出错.
\end{enumerate}

必要的字体文件见表~\ref{tab:postscript}, 可以在这里~\footnote{https://pan.baidu.com/s/1eRFJXnW}下载后放入系统的字体文件夹中.
\begin{table}[htbp]\centering\caption{字体文件}\label{tab:postscript}\begin{tabular}{c|c}
\hline
字体 & PostScript名称\\
\hline
Times New Roman & TimesNewRomanPSMT\\
Arial & ArialMT\\
Courier New & CourierNewPSMT\\
宋体 & SimSun\\
黑体 & SimHei\\
仿宋 & FangSong\\
\hline
\end{tabular}\end{table}
